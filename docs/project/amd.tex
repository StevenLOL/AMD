% Latex header for doxygen 1.8.8
\documentclass[twoside]{book}

% Packages required by doxygen
\usepackage{fixltx2e}
\usepackage{calc}
\usepackage{doxygen}
\usepackage{graphicx}
\usepackage[utf8]{inputenc}
\usepackage{makeidx}
\usepackage{multicol}
\usepackage{multirow}
\PassOptionsToPackage{warn}{textcomp}
\usepackage{textcomp}
\usepackage[nointegrals]{wasysym}
\usepackage[table]{xcolor}

% Font selection
\usepackage[T1]{fontenc}
\usepackage{mathptmx}
\usepackage[scaled=.90]{helvet}
\usepackage{courier}
\usepackage{amssymb}
\usepackage{sectsty}
\renewcommand{\familydefault}{\sfdefault}
\allsectionsfont{%
  \fontseries{bc}\selectfont%
  \color{darkgray}%
}
\renewcommand{\DoxyLabelFont}{%
  \fontseries{bc}\selectfont%
  \color{darkgray}%
}
\newcommand{\+}{\discretionary{\mbox{\scriptsize$\hookleftarrow$}}{}{}}

% Page & text layout
\usepackage{geometry}
\geometry{%
  a4paper,%
  top=2.5cm,%
  bottom=2.5cm,%
  left=2.5cm,%
  right=2.5cm%
}
\tolerance=750
\hfuzz=15pt
\hbadness=750
\setlength{\emergencystretch}{15pt}
\setlength{\parindent}{0cm}
\setlength{\parskip}{0.2cm}
\makeatletter
\renewcommand{\paragraph}{%
  \@startsection{paragraph}{4}{0ex}{-1.0ex}{1.0ex}{%
    \normalfont\normalsize\bfseries\SS@parafont%
  }%
}
\renewcommand{\subparagraph}{%
  \@startsection{subparagraph}{5}{0ex}{-1.0ex}{1.0ex}{%
    \normalfont\normalsize\bfseries\SS@subparafont%
  }%
}
\makeatother

% Headers & footers
\usepackage{fancyhdr}
\pagestyle{fancyplain}
\fancyhead[LE]{\fancyplain{}{\bfseries\thepage}}
\fancyhead[CE]{\fancyplain{}{}}
\fancyhead[RE]{\fancyplain{}{\bfseries\leftmark}}
\fancyhead[LO]{\fancyplain{}{\bfseries\rightmark}}
\fancyhead[CO]{\fancyplain{}{}}
\fancyhead[RO]{\fancyplain{}{\bfseries\thepage}}
\fancyfoot[LE]{\fancyplain{}{}}
\fancyfoot[CE]{\fancyplain{}{}}
\fancyfoot[RE]{\fancyplain{}{\bfseries\scriptsize NLP<Go> Programmer's Manual}}
\fancyfoot[LO]{\fancyplain{}{\bfseries\scriptsize NLP<Go> Programmer's Manual}}
\fancyfoot[CO]{\fancyplain{}{}}
\fancyfoot[RO]{\fancyplain{}{}}
\renewcommand{\footrulewidth}{0.4pt}
\renewcommand{\chaptermark}[1]{%
  \markboth{#1}{}%
}
\renewcommand{\sectionmark}[1]{%
  \markright{\thesection\ #1}%
}

% Indices & bibliography
\usepackage{natbib}
\usepackage[titles]{tocloft}
\setcounter{tocdepth}{2}
\setcounter{secnumdepth}{5}
\makeindex

% Hyperlinks (required, but should be loaded last)
\usepackage{ifpdf}
\ifpdf
  \usepackage[pdftex,pagebackref=true]{hyperref}
\else
  \usepackage[ps2pdf,pagebackref=true]{hyperref}
\fi
\hypersetup{%
  colorlinks=true,%
  linkcolor=blue,%
  citecolor=blue,%
  unicode%
}

% Custom commands
\newcommand{\clearemptydoublepage}{%
  \newpage{\pagestyle{empty}\cleardoublepage}%
}


%===== C O N T E N T S =====

\begin{document}

% Titlepage & ToC
\hypersetup{pageanchor=false,
             bookmarks=true,
             bookmarksnumbered=true,
             pdfencoding=unicode
            }
\pagenumbering{roman}
\begin{titlepage}
\vspace*{7cm}
\begin{center}%
{\Huge NLP<Go>}\\
\vspace*{1cm}
{\huge Programmer's Manual}\\
\vspace*{1cm}
{\large $date}\\
\end{center}
\end{titlepage}
\clearemptydoublepage
\tableofcontents
\clearemptydoublepage
\pagenumbering{arabic}
\hypersetup{pageanchor=true}

%--- Begin generated contents ---


\section{Introduction}
\begin{frame}
\frametitle{Covariance: A Motivating Example}
%

\begin{center}
\colorbox{green!10}{
$\textnormal{Gaussian pdf} = 
    \color{m1}(2\pi{})^{\!-\frac{n}{2}}\det(\mSigma)^{\!-\frac{1}{2}}
           e^{(\mx-\mu)^\top{}\mSigma^{-1}(\mx-\mu)}$
}
\end{center}
%
Tikhonov regularized maximum log-likelihood for learning the covariance
$\color{m1}\mSigma$ given an empirical covariance $\color{m1}\mS$ is:
%
\begin{center}
%
\colorbox{green!10}{
%
$\color{m1}
f(\mSigma)=-\logdet(\mSigma)-\trace(\mS\mSigma^{-1})-\|\mSigma^{-1}\|_F^2.$
%
}
%
\end{center}
%
\begin{itemize}
%
\item To optimize $\color{m1}f(\mSigma)$, using first-order methods we need 
      $\color{m1}f'(\mSigma)$.
%
\item To use second-order methods, we also need $\color{m1}f''(\mSigma)$.
%
\end{itemize}
%
\begin{center}
%
What are $\color{m1}f'(\mSigma)$ and $\color{m1}f''(\mSigma)$?
%
\end{center}
%
\end{frame}

%
\begin{frame}
\frametitle{Objective}
\framesubtitle{Theoretical Aspect}
%
\begin{center}
%
Differentiation of matrix functions is still not well understood.
%
\end{center}
%
\begin{itemize}
%
\item No simple ``calculus'' for computing matrix derivatives.
%
\item Matrix derivatives of simple functions can be complicated.
%
\end{itemize}
%
\begin{center}
%
\textcolor{blue}{Build on known results to define a matrix-derivative
calculus.}
%
\end{center}
%
\end{frame}

\begin{frame}
\frametitle{Objective}
\framesubtitle{Software Aspect}
%
\begin{center}
\textcolor{blue}{Matrix differentiation made simple for humans and computers}
\end{center}
%
\begin{itemize}
\item Symbolic differentiation should be made simple.
\item Optimizing matrix valued functions should be efficient:
  \begin{itemize}
  \item Both computation and storage.
  \end{itemize}
\end{itemize}

%
\begin{center}
We are developing Automatic Matrix Differentiation (AMD), an open-source \Cpp{}
library to accomplish this.  The code can be downloaded from
\textcolor{blue}{\url{https://github.com/pkambadu/AMD}}
\end{center}
\end{frame}

\begin{frame}
\frametitle{Description}
\begin{center}
\textcolor{blue}{Re-design AMD from ground up}
\end{center}

\begin{itemize}
\item Previous implementation led to key insights.
\item Rewriting from scratch better than fixing existing design.
    \begin{itemize}
    \item Easy creation of matrix-matrix functions.
    \item Optimization of abstract syntax trees.
    \item Advanced logging and exception handling.
    \item Parallel evaluation on several matrix types.
    \item Python bindings.
    \end{itemize}
\end{itemize}
\end{frame}

\begin{frame}[fragile]
\frametitle{Expected \Cpp{} Interface}

\begin{lstlisting}[style=basic]
@{\color{red} /** Associate matrices with symbols */}@
std::map<std::string, boost::shared_ptr<MatrixType> > symbolTable;
symbolTable["A"] = ...; symbolTable["X"] = ...;

@{\color{red} /** Create an expression and evaluate it */}@
AMD::Expression fX("trace(A*(I+X)");
AMD::evaluate(fX, symbolTable);

@{\color{red} /** Evaluate the derivative */}@
AMD::Expression fX_prime = AMD::gradient(fX);
AMD::evaluate(fX_prime, symbolTable);
\end{lstlisting}
\end{frame}

\begin{frame}
\frametitle{Creating Expression Trees}
%
\begin{center}
\textcolor{blue}{Accept a well-formed matrix op-string and return an AST"}
\end{center}
%
\begin{itemize}
\item Use \texttt{Boost.Spirit}.
\item Matrix-matrix functions: +, -, *, .*, ',\textunderscore{}
\item Scalar-Matrix functions: \code{trace}, \code{logdet}.
\item Obey precedence and associativity (include parantheses).
\end{itemize}
%
\end{frame}

\begin{frame}
\frametitle{Optimizing Expression Trees}
%
\begin{center}
\textcolor{blue}{Reduce computational complexity by rewriting the AST}
\end{center}
%
\begin{itemize}
\item Use all known matrix identities.
\item \code{(A')' = A}, \code{(A_)_ = A}
\item \code{I*A = A*I = A}, \code{0+A = A+0 = A}
\item Common sub-expression elimination
\item $\ldots{}$
\end{itemize}
%
\end{frame}

\begin{frame}
\frametitle{Derivatives Of Scalar-Matrix Functions}
%
\begin{center}
\textcolor{blue}{Compute the derivative using reverse mode}
\end{center}
%
\begin{itemize}
\item Apply the rules from the AMD paper.
\item Apply optimizations post-derivation.
\end{itemize}
%
\end{frame}

\begin{frame}
\frametitle{Evaluation Expression Trees}
%
\begin{center}
\textcolor{blue}{Evaluate the AST in parallel}
\end{center}
%
\begin{itemize}
\item Identify parallelism (topological sort).
\item Evaluation on a symbol table.
\end{itemize}
%
\end{frame}


\begin{frame}
\frametitle{Acknowledgements}
\begin{itemize}
\item Bloomberg: Keti, Rohan, Shefaet, James, Tom, Kevin, $\dots{}$.
\item IBM Research: Steve Rennie.
\item Home: Melanie Kambadur.
\end{itemize}
%
\begin{center}
\colorbox{green!10}{
{\texttt{\color{m1}Thanks For Your Attention}}
}
\end{center}
%
\end{frame}

\end{document}

