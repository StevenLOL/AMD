\section{A Taylor Expansion}

\begin{frame}
\frametitle{Taylor Expansion}
Recall the Taylor expansion of a function $\color{m1}f$ around a value
$\color{m1}x_0$:
%
\footnotesize
$$\color{m1}
f(x) = f(x_0)+(x-x_0)f'(x_0)+\frac{1}{2}(x-x_0)^2f''(x_0) + 
       \frac{1}{3!}(x-x_0)^3f'''(x_0)+\cdots
$$
\normalsize
%
For a vector valued function $\color{m1}f'(\vx_0)$:
%
\footnotesize
$$\color{m1}
f(\vx) = f(\vx_0)+(\vx-\vx_0)^{\top}f'(\vx_0) + 
         \frac{1}{2}(\vx-\vx_0)^{\top} f''(\vx_0)(\vx-\vx_0)+\cdots
$$
\normalsize
\vspace{-10pt}
\begin{itemize}
\item $\color{m1}f''(\vx_0)$ is a matrix.
\item $\color{m1}f'''(\vx_0)$ is a 3-D tensor.
\begin{itemize}
\item It is unclear how to continue the expansion.
\end{itemize}
\end{itemize}
\end{frame}

\begin{frame}
\frametitle{Taylor Expansion}
\framesubtitle{Scalar-matrix Functions: $\trace\mF(\mX)$}
%
Let $\color{m1} F(\mX)=(\mI-\mX)^{-1}$; expanding around $\color{m1}\mX_0=0$ 
we get: 
%we can pretend $\color{m1}\mX$ is a scalar and get
%
$$\color{m1}
(\mI-\mX)^{-1}= \mI+\mX+\mX^2+\mX^3+\cdots
$$
%
So, $\color{m1}f(\mX)=\trace((\mI-\mX)^{-1})$ we can write the Taylor expansion
as:
%\footnotesize
$$\color{m1}
\trace((\mI-\mX)^{-1})=
\trace(\mI)+\trace(\mX)+\trace(\mX^2)+\trace(\mX^3)+\cdot,
$$
%\normalsize
%
\vspace{-10pt}
\begin{itemize}
\item We do not compute $\color{m1}f'(\mX), f''(\mX), f'''(\mX),\cdots$ 
      explicity.
\item We can compute $\color{m1}f''(\mX)vec(\mX)$ efficiently using our code.
\item Automate the process of building the Taylor expansion.
\end{itemize}
\end{frame}

%\begin{frame}[fragile]
%\frametitle{Taylor Expansion}
%\framesubtitle{$\logdet(\mX)$}
%We use these matrix differentiation rules to compute the Taylor expansion
%around the point $\color{m1}\mX_0$.  Our code gives the following Taylor
%expansion for $\color{m1}\logdet(\mX)$ as seen in the code on the next two
%slides.
%\begin{equation*}
%\begin{split}
%\color{m1} \logdet(\mX) &= \color{m1}\logdet(\mX_0) + \trace((\mX-\mX_0)\mX_0^{-1}) \\
%         &\color{m1}+\frac{1}{2}\trace(-(\mX-\mX_0)\mX_0^{-1}(\mX-\mX_0)\mX_0^{-1}) \\
%         &\color{m1}+ \frac{1}{3}\trace((\mX-\mX_0)\mX^{-1}(\mX-\mX_0)\mX^{-1}(\mX-\mX_0)\mX^{-1})+\cdots
%\end{split}
%\end{equation*}
%\end{frame}

\begin{frame}
\frametitle{Taylor Expansion}
\framesubtitle{$\logdet(\mX)$ From Algebra}
%
Let $\color{m1}\mP=\mP_0+\mDelta$ be a symmetrix matrix;
$\color{m1}\mX=\mP_0^{-\frac{1}{2}}\mDelta\mP_0^{-\frac{1}{2}}$.
%
%
\onslide<2->{
Let, $\color{m1}\mQ^\top\mE\mQ$ be the eigen-decomposition of $\color{m1}\mX$.
}
\footnotesize
%
{\color{m1}
\begin{eqnarray*} 
\logdet \mP &=& \logdet(\mP_0+\mDelta)\\ 
\onslide*<1>{
&\color<1>[rgb]{1,0,0}{=}&  
\color<1>[rgb]{1,0,0}{\logdet(\mP_0^{1/2}(\mI+\mP_0^{-1/2}\mDelta \mP_0^{-1/2})\mP_0^{1/2})}\\
&\color<1>[rgb]{1,0,0}{=}&  
\color<1>[rgb]{1,0,0}{\logdet\mP_0+\logdet(\mI+\mP_0^{-1/2}\mDelta
  \mP_0^{-1/2})}\\
} % onslide*<1>
\onslide*<1-2>{&=&\logdet\mP_0+\logdet(\mI+\mX)\\} 
\onslide*<2>{
&\color<2>[rgb]{1,0,0}{=}&
\color<2>[rgb]{1,0,0}{\logdet\mP_0+\logdet(\mQ^\top\mQ+\mQ^\top\mE\mQ)}\\
&\color<2>[rgb]{1,0,0}{=}&
\color<2>[rgb]{1,0,0}{\logdet\mP_0+\logdet(\mQ^\top(\mI+\mE)\mQ)}\\
&\color<2>[rgb]{1,0,0}{=}&\color<2>[rgb]{1,0,0}{\logdet\mP_0+\logdet(\mI+\mE)}\\}% slide 2
\onslide*<2->{
&=& \logdet \mP_0+\sum_{i=1}^d\log(1+e_i)\\ }
\onslide*<3>{
&\color<3>[rgb]{1,0,0}{=}& \color<3>[rgb]{1,0,0}{\logdet
\mP_0+\sum_{i=1}^d\sum_{k=1}^\infty\frac{(-1)^{k+1}}{k}e_i^k}\\
&\color<3>[rgb]{1,0,0}{=}& \color<3>[rgb]{1,0,0}{\logdet
\mP_0+\sum_{k=1}^\infty\frac{(-1)^{k+1}}{k}\sum_{i=1}^de_i^k}\\}
\onslide*<3>{
&\color<3>[rgb]{1,0,0}{=}& \color<3>[rgb]{1,0,0}{\logdet
\mP_0+\sum_{k=1}^\infty\frac{(-1)^{k+1}}{k}\trace(\mE^k)}\\}
\end{eqnarray*}}
\normalsize
\end{frame}

\begin{frame}
\frametitle{Taylor Expansion}
\framesubtitle{Convenience: $\trace(\mE^k) = \trace(\mX^k)$:}
%
When, $\color{m1}\mX=\mQ^\top\mE\mQ$, we have $\color{m1}\trace(\mE^k) =
\trace(\mX^k)$:
%
{\color{m1}
\begin{align*}
\trace(\mE^k) &= \trace(\mE\mE\cdots\mE) \\
&= \trace(\mE\mQ\mQ^\top\mE\mQ\mQ^\top\cdots\mE\mQ\mQ^\top) \\
&= \trace(\mQ^\top\mE\mQ\mQ^\top\mE\mQ\mQ^\top\cdots\mE\mQ) \\
&= \trace((\mQ^\top\mE\mQ)^k) \\
&= \trace(\mX^k)
\end{align*}
}
%
We use the following:
\begin{itemize}
\item $\color{m1}\mQ^\top\mQ = \mQ\mQ^\top = \mI$ when $\color{m1}\mX$ is
      symmetric.
\item $\color{m1}\trace(\cdot)$ is invariant under unitary transforms.
\end{itemize}
\end{frame}

\begin{frame}
\frametitle{Taylor Expansion}
\framesubtitle{Scalar-Matrix Functions: $\logdet\mP$}
\footnotesize
{\color{m1}
\begin{align*}
\logdet \mP 
&= \logdet \mP_0+\sum_{k=1}^\infty\frac{(-1)^{k+1}}{k}\trace(\mE^k) \\
&= \logdet \mP_0+\sum_{k=1}^\infty\frac{(-1)^{k+1}}{k}\trace(\mX^k) \\
&= \logdet(\mP_0)+\trace(\mX)-1/2\trace(\mX^2)+\cdots \\
&= \logdet(\mP_0)+\trace(\mDelta\mP_0^{-1}) - 
  1/2\trace(\mDelta\mP_0^{-1}\mDelta\mP_0^{-1})+\cdots
\end{align*}
}
\normalsize
%
Notice that:
\footnotesize
{\color{m1}
\begin{align*}
\trace(\mDelta\mP_0^{-1}) &= 
            \trace\left(\mDelta
           {\color{m2}
 \left(\frac{\partial}{\partial\mP_0}(\logdet\mP_0)\right)^\top}\right) \\
           %{\color{m2}\left(\mP_0^{\!-\top}\right)}^\top \right) \\
            %\myvec^\top(\mDelta)\myvec(\mP_0^{-1}) \\
\trace(\mDelta\mP_0^{-1}\mDelta\mP_0^{-1}) &= 
      \trace\left( \mDelta 
          {\color{m2}\left( 
              \frac{\partial}{\partial\mP_0}
              \left(\trace(\Delta\mP_0^{-1})\right)\right)^\top}\right) \\
    %{\color{m2}\left(\mP_0^{\!-\top}\otimes\mP_0^{\!-1}\right)}^\top
    %               \left(\mI\otimes\mDelta\right) \right)  \\
%\myvec^\top(\mDelta)(\mP_0^{-1}\otimes\mP_0^{-1})\myvec(\mDelta) \\
\end{align*}
}
\normalsize
%
\end{frame}

\begin{frame}[fragile]
\frametitle{Taylor Expansion}
\framesubtitle{$\logdet(\mX)$}
If we define $\color{m1}\mDelta=(\mX-\mX_0)$ to be a constant function (with
respect to $\color{m1}\mX_0$) we can compute the Taylor series iteratively as
follows:
{\color{m1}
\begin{align*}
f_0(\mX_0) &= \logdet(\mX_0) \\
f_1(\mX_0) &= \trace(\mDelta(f_0'(\mX_0))^{T})\\
f_2(\mX_0) &= \trace(\mDelta(f_1'(\mX_0))^{T})\\
f_3(\mX_0) &= \trace(\mDelta(f_2'(\mX_0))^{T})
\end{align*}
}
The Taylor expansion can now be written:
$$
\color{m1} \logdet(\mX) = f_0(\mX_0) + f_1(\mX_0) + \frac{1}{2!}f_2(\mX_0) +
\frac{1}{3!}f_3(\mX_0) \cdots
$$
\end{frame}

\begin{frame}[fragile]
\frametitle{Taylor expansion for $\logdet(\mX)$ continued} 
\tiny
\begin{lstlisting}[style=basic]
  @\alert{/* Initialize matrices */}@
  SymbolicMatrixMatlab X("X",M,M), X0("X0",M,M), Delta("(X-X0)",M,M);
  SymbolicScalarMatlab a2("1/2!"), a3("1/3!");
  @\alert{/* Define Scalar-Matrix and Matrix-Matrix functions */}@
  SymbolicSMFunc r2(a2,M,M), r3(a3,M, M);
  SymbolicMMFunc fX(X, false), fX0(X0, false), fDelta(Delta, true);
  @\alert{/* Compute Taylor series terms iteratively */} @
  SymbolicSMFunc f0 =  logdet(fX0);
  SymbolicSMFunc f1 = trace(fDelta * transpose(*f0.derivativeFuncVal));
  SymbolicSMFunc f2 = trace(fDelta * transpose(*f1.derivativeFuncVal));
  SymbolicSMFunc f3 = trace(fDelta * transpose(*f2.derivativeFuncVal));
  @\alert{/* Taylor Expansion with four terms */}@
  SymbolicSMFunc func = f0 + f1 + r2*f2 + r3*f3;
\end{lstlisting}
\normalsize
\end{frame}

\begin{frame}[fragile]
Here is the result:
\scriptsize
\begin{semiverbatim}
./TaylorSample.exe
The first 4 terms of Taylor Expansion for logdet(X) around X0 is:
(((log(det(X0))
 +trace((X-X0)*inv(X0)))
 +(1/2!*trace((X-X0)*(-((inv(X0)*(X-X0))*inv(X0))))))
 +(1/3!*trace((X-X0)*( (-((inv(X0)*((X-X0)*(inv(X0)*(-(X-X0)))))*inv(X0)))'
             +(-((inv(X0)*((-(X-X0))*(inv(X0)*(X-X0))))*inv(X0)))')')))
\end{semiverbatim}
\end{frame}
