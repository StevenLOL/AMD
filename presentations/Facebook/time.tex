\section{How We Spent Our Time}

\begin{frame}
\frametitle{Learning Terminology}
Classical calculus:
\begin{itemize}
\item \alert{scalar-scalar functions} ($\color{m1}f:\R\rightarrow\R$).
\item \alert{scalar-vector functions} ($\color{m1}f:\R^d\rightarrow\R$).  
\end{itemize}

Matrix calculus:
\begin{itemize}
\item \alert{scalar-matrix functions}
        ($\color{m1}f:\R^{m\times{}n}\rightarrow\R$).
\item \alert{matrix-matrix functions}
        ($\color{m1}f:\R^{m\times{} n}\rightarrow\R^{k\times l}$).
\end{itemize}

Note:
\begin{enumerate}
\item Derivatives of scalar-matrix functions require derivatives of
matrix-matrix functions. 
\item Matrix-matrix derivatives should be computed implicitly when possible.
\end{enumerate}
\end{frame}

\begin{frame}
\frametitle{Learning Numerical Techniques}
%
\begin{description}
\item[Numerical] Use finite difference formulae for differentiation (e.g.,
$\color{m1} f'(x) \approx \frac{f(x+h)-f(x)}{h} $).
%
These methods are simple to program, but loose half of all significant
digits.
%
\item[Symbolic] Give the symbolic representation of the derivative without
saying how best to implement the derivative; these are as accurate as they
come.
%
\item[Algorithmic]  Something between numeric and symbolic differentiation.
The derivative implementation is computed from the code for $\color{m1}f(x)$.
%
\end{description}
%
\end{frame}

\begin{frame}
\frametitle{Learning Automatic Differentiation}
%
Consider a concrete $\color{m1}f(x)$ and it's transformation:
%
\tiny
\begin{eqnarray*}
  \color{m1}f(x)= (x-1)(x-2)(x-3) \\
  \color{m1}    = *_2(*_1(x-1,x-2),x-3) \\
  \color{m1}f'(x) = \frac{d*_2}{d*_2}
                      \left[
                        \frac{\partial *_2}{\partial *_1}
                          \left(
                            \frac{\partial *_1}{\partial (x-1})\frac{d(x-1)}{dx} +
                            \frac{\partial *_1}{\partial (x-2)}\frac{d(x-2)}{dx}
                          \right) +
                          \frac{\partial *_2}{\partial (x-3)}\frac{d(x-3)}{dx}
                    \right] \\
  \color{m1} = (x-2)(x-3)+(x-1)(x-3)+(x-1)(x-2) 
\end{eqnarray*}
\normalsize
%
\vspace{-10pt}
%
\begin{itemize}
%
  \item Apply the chain rule repeatedly to obtain $\color{m1}f'(x)$.
  \item Choices:
    \begin{itemize}
      \item Inside-out application (forward-mode)
      \item Outside-in application (reverse-mode)
    \end{itemize}
%
\end{itemize}
%
\end{frame}

\begin{frame}
\frametitle{Learning Rules of Matrix-Matrix Derivative}
We define the matrix--matrix derivative to be: 
%
$$\color{m1}
\frac{\partial \mF}{\partial \mX} \stackrel{\mathrm{def}}{=} 
\frac{\partial
  \myvec(\mF^\top)}{\partial \myvec^\top(\mX^\top)} = 
\begin{pmatrix} 
\frac{\partial f_{11}}{\partial x_{11}} &
\frac{\partial f_{11}}{\partial x_{12}} & \cdots & 
\frac{\partial f_{11}}{\partial x_{mn}}\\
\frac{\partial f_{12}}{\partial x_{11}} &
\frac{\partial f_{12}}{\partial x_{12}} & \cdots & 
\frac{\partial f_{12}}{\partial x_{mn}}\\
\vdots & \vdots & \ddots & \vdots \\
\frac{\partial f_{kl}}{\partial x_{11}} &
\frac{\partial f_{kl}}{\partial x_{12}} & \cdots & 
\frac{\partial f_{kl}}{\partial x_{mn}}
\end{pmatrix}.
$$
%
Note that the matrix--matrix derivative of a scalar--matrix function
is not the same as the scalar--matrix derivative:
$$\color{m1}
\frac{\partial\;\mbox{mat}(f)}{\partial\mX} = \myvec^\top\left(
\left( \frac{\partial f}{\partial\mX}\right)^\top
\right) .
$$
\end{frame}
